\documentclass[14pt, a4paper]{extarticle} 
\usepackage[T2A]{fontenc}
\usepackage[utf8]{inputenc}
\usepackage[english, russian]{babel}
\usepackage[left=1cm,right=1cm,top=2cm,bottom=2cm]{geometry}

\begin{document}

\begin{flushright}
  Кудашева Милана, 151 гр.
\end{flushright}
\begin{center}
    \Large \textbf{Глоссарий}
\end{center}

\begin{enumerate}
    \item \textbf{Agile Modeling} "--- перечень принципов, терминов и практик, использование которых ускоряет и упрощает разработку моделей ПО и документацию. \textit{(Никита Барабанов)}
    \item \textbf{AlphaZero} "--- нейронная сеть, разработанная компанией DeepMind, которая использует обобщённый подход AlphaGo Zero. \textit{(Алексей Кузьмин)}
    \item \textbf{Angular} "--- фреймворк от компании Google для создания продвинутых бесшовных веб"=приложений. \textit{(Алексей Кузьмин)}
    \item \textbf{Bard} "--- чат бот, разработанный компанией Google. Как и другие подобные системы, он использует нейросети для создания ответов на запросы пользователей. \textit{(Никита Рыданов)}
    \item \textbf{Big Data} "--- структурированные или неструктурированные массивы данных большого объема. \textit{(Алексей Кузьмин)}
    \item \textbf{Business intelligence} "--- обозначение компьютерных методов и инструментов для организаций, обеспечивающих перевод транзакционной деловой информации в человекочитаемую форму, а также средства для массовой работы с такой обработанной информацией. \textit{(Алексей Кузьмин)}
    \item \textbf{Data Mining} "--- способ анализа данных, предназначенный для поиска ранее неизвестных закономерностей в больших массивах информации. \textit{(Алексей Кузьмин)}
    \item \textbf{Data Science} "--- междисциплинарная область на стыке статистики, математики, системного анализа и машинного обучения, которая охватывает все этапы работы с данными. \textit{(Алексей Кузьмин)}
    \item \textbf{Deep Learning} "--- тип машинного обучения, который с помощью искусственных нейронных сетей обеспечивает цифровым системам возможность обучаться и принимать решения на основе неструктурированных данных без меток. \textit{(Алексей Кузьмин)}
    \item \textbf{DevOps} (акр. от <<Development Operations>>) "--- методология разработки, которая помогает наладить эффективное взаимодействие разработчиков с другими IT"=специалистами. \textit{(Алексей Кузьмин)}
    \item \textbf{ERP"=система} "--- набор интегрированных приложений или модулей для управления основными бизнес"=процессами компании, включая финансы и бухгалтерский учет, цепочку поставок, управление персоналом, закупки, продажи, управление запасами и многое другое. \textit{(Ростислав, <<Сибинтек>>)}
    \item \textbf{Fast Data}  "--- совокупность технологий, обеспечивающих скоростную обработку потоков данных, генерируемых различными системами и активностями, их анализ и принятие решений с учетом исторически накопленных данных, а также автоматическое выполнение различных действий в ответ на произошедшие события. \textit{(Алексей Кузьмин)}
    \item \textbf{FastAPI} "--- современный, быстрый (высокопроизводительный) веб"=фреймворк для создания API используя Python 3.6+, в основе которого лежит стандартная аннотация типов Python. \textit{(Алексей Кузьмин)}
    \item \textbf{GAN} (акр. от <<Generative Adversarial Network>>) "--- алгоритм машинного обучения без учителя, построенный на комбинации из двух нейронных сетей, одна из которых генерирует образцы, а другая старается отличить правильные (<<подлинные>>) образцы от ложных. \textit{(Алексей Кузьмин)}
    \item \textbf{GPT} (акр. от <<Generative Pre"=trained Transformer>>) "--- тип нейронных языковых моделей, впервые представленных компанией OpenAI, которые обучаются на больших наборах текстовых данных, чтобы генерировать текст, схожий с человеческим. \textit{(Михаил Чернигин)}
    \item \textbf{MLOps} (акр. от <<Machine Learning Operations>>) "--- DevOps"=подход для создания решений с машинным обучением. \textit{(Алексей Кузьмин)}
    \item \textbf{Open Source} "--- исходный код программного обеспечения, который доступен для всех пользователей. \textit{(Никита Рыданов)}
    \item \textbf{QA"=инженер} (акр. от <<Quality Assurance Engineer>>) "--- специалист по обеспечению качества разработки ПО. \textit{(Павел, <<Тинькофф>>)}
    \item \textbf{Roadmap} "--- документ, в котором перечислены цели проекта, его ключевые этапы, контрольные даты и ответственные исполнители. \textit{(Павел, <<Тинькофф>>)}
    \item \textbf{SOA} (акр. от <<Service"=Oriented Architecture>>) "--- метод разработки программного обеспечения, который использует программные компоненты, называемые сервисами, для создания бизнес"=приложений. \textit{(Никита Барабанов)}
    \item \textbf{SQL} (акр. от <<Structured Query Language>>) "---декларативный язык программирования (язык запросов), который используют для создания, обработки и хранения данных в реляционных БД. \textit{(Иван Жадаев)}
    \item \textbf{UX/UI"=дизайнер} "--- профессионал, занимающийся разработкой и внедрением простых и понятных интуитивных интерфейсов в новые или действующие цифровые продукты. \textit{(Алексей Кузьмин)}
    \item \textbf{Waterfall} "--- каскадная модель управления проектами, при которой происходит последовательный переход с одного этапа на другой, при этом пропуск отдельного этапа и возврат на предыдущие стадии не предусмотрен. \textit{(Никита Барабанов)}
    \item \textbf{Админ"=панель} "--- инструмент для управления веб"=ресурсом и его настройками, добавления новых и удаления старых страниц, изменения внешнего вида веб"=ресурса и редактирования контента. \textit{(Иван Жадаев)}
    \item \textbf{Аппроксимация} "--- научный метод, состоящий в замене одних объектов другими, в каком"=то смысле близкими к исходным, но более простыми. \textit{(Алексей Кузьмин)}
    \item \textbf{Баг} (англ. bug, сленг) "--- ошибка в коде. \textit{(Екатерина, <<Сибинтек>>)}
    \item \textbf{Геймдев} (акр. от <<Games Development>>) "--- процесс создания игры: от разработки и дизайна до выпуска на рынок. Это могут быть игры для мобильных телефонов, консолей, компьютеров или других гаджетов. \textit{(Никита Барабанов)}
    \item \textbf{Граф} "--- абстрактный тип данных, который предназначен для реализации понятий неориентированного графа и ориентированного графа из области теории графов в математике. \textit{(Иван Жадаев)}
    \item \textbf{Дискретизация} "--- преобразование непрерывных изображений и звука в набор дискретных значений в форме кодов. \textit{(Алексей Кузьмин)}
    \item \textbf{Запушить} (англ. push, сленг) "--- отправить код в репозиторий. \textit{(Никита Барабанов)}
    \item \textbf{Интерфейс} "--- программная/синтаксическая структура, определяющая отношение между объектами, которые разделяют определённое поведенческое множество и не связаны никак иначе. \textit{(Павел, <<Тинькофф>>)}
    \item \textbf{Канбан} "--- японская система оптимизации и управления проектами и производством. \textit{(Никита Барабанов)}
    \item \textbf{Кластерный анализ} "--- разделение большой группы объектов на несколько поменьше. \textit{(Алексей Кузьмин)}
    \item \textbf{Матрица} "--- обобщенный термин для различных объектов в электронике, в которых элементы объекта упорядочены в виде двумерного массива, аналогично математической матрице. \textit{(Михаил Чернигин)}
    \item \textbf{Машинное обучение} "--- использование математических моделей данных, которые помогают компьютеру обучаться без непосредственных инструкций. \textit{(Алексей Кузьмин)}
    \item \textbf{Микросервис} "--- веб"=сервис, отвечающий за один элемент логики в определенной предметной области. \textit{(Алексей Кузьмин)}
    \item \textbf{Монолитное приложение} "--- приложение, которое состоит из одной большой кодовой базы, которая включает в себя все компоненты: код фронтенда, код бэкенда и файлы конфигурации. \textit{(Никита Барабанов)}
    \item \textbf{Облачные вычисления} "--- технология, которая обеспечивает доступ к компьютерным ресурсам через интернет. \textit{(Алексей Кузьмин)}
    \item \textbf{Паттерн проектирования} "--- повторяемая архитектурная конструкция, применяемая для решения часто встречающихся задач. \textit{(Иван Жадаев)}
    \item \textbf{Пирамида тестирования} "--- метафора, представляющая собой пирамиду, состоящую из разного уровня тестов: модульных, интеграционных, пользовательских. \textit{(Павел, <<Тинькофф>>)}
    \item \textbf{Профит} (англ. profit, сленг) "--- доход, прибыль, выгода, польза. \textit{(Никита Барабанов)}
    \item \textbf{Ранжирование} "--- класс задач машинного обучения с учителем, заключающихся в автоматическом подборе ранжирующей модели по обучающей выборке, состоящей из множества списков и заданных частичных порядков на элементах внутри каждого списка. \textit{(Алексей Кузьмин)}
    \item \textbf{Регрессионный анализ} "--- статистический метод, устанавливающий количественно форму зависимости двух случайных величин, между которыми существует корреляционная связь. \textit{(Алексей Кузьмин)}
    \item \textbf{Сверточные сети} "--- специализированный тип нейронных сетей, которые используют свертку вместо общего матричного умножения по крайней мере в одном из своих слоев. \textit{(Алексей Кузьмин)}
    \item \textbf{Система управления базами данных} "--- набор программ, которые управляют структурой БД и контролируют доступ к данным, хранящимся в БД. \textit{(Иван Жадаев)}
    \item \textbf{Спринт} (англ. sprint, сленг) "--- небольшой фиксированный отрезок времени, в который команда делает какую"=то ограниченную часть проекта. \textit{(Екатерина, <<Сибинтек>>)}
    \item \textbf{Тимлид} (англ. team leader, сленг) "--- IT"=специалист, который управляет командой разработчиков, владеет технической стороной, принимает участие в работе над архитектурой проекта, занимается проверкой кода, а также разработкой некоторых особо сложных заданий на проекте. \textit{(Павел, <<Тинькофф>>)}
    \item \textbf{Трансформер} "--- архитектура глубоких нейронных сетей, представленная в 2017 году исследователями из Google Brain. \textit{(Никита Рыданов)}
    \item \textbf{Фулстек разработчик} "--- программист, который отвечает за все этапы разработки сайта или приложения. Эти специалисты создают сайты на языке программирования JavaScript. Они работают с визуальной и серверной частями веб"=сервиса. \textit{(Павел, <<Тинькофф>>)}
    \item \textbf{Эвристика} "--- совокупность приёмов и методов, облегчающих и упрощающих решение познавательных, конструктивных, практических задач. \textit{(Алексей Кузьмин)}
\end{enumerate}

\end{document}